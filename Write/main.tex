\documentclass{article}
\usepackage[english]{babel}
\usepackage[utf8]{inputenc}
\usepackage{amsmath}
\usepackage{amssymb}
\usepackage{algorithm}% http://ctan.org/pkg/algorithms
\usepackage{algpseudocode}% http://ctan.org/pkg/algorithmicx
\usepackage{tikz-cd}
\usepackage[colorlinks, linkcolor = blue, citecolor = magenta]{hyperref}
\title{TenSyGrid}

\author{Pablo de Juan Vela $^{1}$ \\
        \small $^{1}$eRoots, Barcelona, Spain \\
}
\date{}

\begin{document}
\maketitle
\begin{abstract} 
\end{abstract}

\section{Tensor Notation}

A tensor is an N-way array where its values are accesed by N indeces. In this sense, a vector is an order 1 tensor and a matrix an order 2 tensor.
A tensor of order n in the real numbers can de defined as $F \in \mathbb{R}^{I_1 \times ... \times I_n}$, where $I_i$ is the dimension of the $i_{th}$ order.



\subsection{Tensor Operations}

Let $F \in \mathbb{\mathbb{R}^{i_1 \times ... \times i_n}$. Then the tensor product can be defined as 

\begin{equation}
    F \otimes G :=
\end{equation}

\begin{equation}
    <F,G>
\end{equation}


\section{Multilinear models}

A multi linear equation takes the form of:

\begin{align}
    p(x) &= \sum_{i_n = 0}^{i_n = 1}...\sum_{i_1 = 0}^{i_1 = 1} a_{i_1, ..., i_n} x_1^{i_1}...x_n^{i_n} \\ 
\end{align}

We can stablish a relationship between the coefficients of a multilinear polinomial in $x\mathbb{R}^{n}$ 
and the entries of a tensor $F \in \mathbb{R}^{\times_2 n}$. The following equation describes the relationship between
the multi linear polynomial $p(x)$ and a tensor $F$.

\begin{equation}
    F_{i_1,...,i_n} = a_{i_1, ..., i_n}
\end{equation}



\subsection{Multilinear Time Independent Models}

Multilinear Time Independent models or (MTI) describes a system and its evolution over time. 
The values used to define the system's operations are the following:

\begin{itemize}
    \item $x \in\mathbb{R}^n$ the state.   
    \item $y \in\mathbb{R}^p$ the output.   
    \item $u \in\mathbb{R}^m$ the input.   
\end{itemize}

Additionally, we need to consider the system's equations. 
Using the previous tensor notation we can describe the system equations explicitly as follows,
where $F \in \mathbb{R}^{\times_2 (n+m) \times n}$, $G \in \mathbb{R}^{\times_2 (n+m) \times p}$:

\begin{align}
    \dot{x} &= <F|M(x,u)>
    y &= <G|M(x,u)>
\end{align}

Another way to define an MTI's state evolution is through the use of an implicit multilinear equation.
In this case the tensor $H \in \mathbb{R}^{\times_2 (2n+m+p) \times n+p}$ defines the state evolution. 
Additionnally, we would need to add the conditions $det(\partial_{\dot{x}} H) \neq 0$, $det(\partial_y H) \neq 0$ 
for the system to be properly defined (implicit function theorem). Otherwise, more equations would be need to 
define the system completely.

\begin{equation}
    <H| M(\dot{x}, x, u, y)> = 0
\end{equation}

It is clear that the explicit formulation can be transformed into the implicit one such that the resulting system is:
Where $\overline{H}$ is chosen approprietely to represent the $(n+p)$ multilinear equations $\dot{x} - <F|M(x,u)> = 0$
and $y - <G|M(x,u)> = 0$.
\begin{equation}
    <\overline{H}| M(\dot{x}, x, u, y)> = 0
\end{equation}

\section{Numerical Methods}

\subsection{Numerical Methods for ODE's}

ODEs can be represented by the following implicit equation:

\begin{equation}\label{ODE:implicit}
    f(x, \dot{x}, t) = 0
\end{equation}

One specific method to solve a ystem of ODE are implicit methods. 
This is a type of family of methods that discretizes the time variable into a set of timesteps
$\mathcal{T}:=\{1,...,T\}$. An initial value $x_0$ for $x$ is given and then the following values are 
found by solving an equation involving the function in \ref{ODE:implicit}. 
More specifically, we solve the equation $g(x_{t+1}, x_t) = 0$ where $x_t$ is known, $x_{t+1}$ is the unknown,
anf $g$ has the following expression:

\begin{equation}
    g(x_{t+1}, x_t) = f(x_{t+1}, \frac{x_{t+1}- x_t}{\Delta t}, t+1) 
\end{equation}

We want to adapt this method to solve an explicit DAE as in \ref{DAE:explicit}. To do so we 
define a new function $\overline{F}(x,u)$ that can be used as the implicit function. 

\begin{equation}
    \overline{F}(\dot{x}, x, z, u) =
    \begin{pmatrix}
        \dot{x} - <F|M(x, z, u)> \\
        <G|M(x, z, u)>
    \end{pmatrix} 
\end{equation}

\begin{equation}
    \overline{G}(x_{t+1}, x_{t}, z_{t+1}, u_{t+1}) =  \overline{F}(\frac{x_{t+1}- x_t}{\Delta t}, x_{t}, z_{t+1}, u_{t+1})
\end{equation}

We can describe the method in algorithmic form as follows:



\section{Case Study 1}

For this case study we define a non linear EDO of the following form:
Where $x \in \mathcal{C}(\mathbb{R}^n, \mathbb{R})$, $A, B, C \in \mathbb{R}^{n \times n } $.
\begin{equation}
    \dot{x}(t) = Ax(t) + B(x(t) \odot x(t)) + x^T(t) C x(t)
\end{equation}

The equation represents an EDO with a linear component $A$ and a quadratic component $B$.
The objective is to transform this ODE into a multi linear DAE.
For this purpose, we add $n$ auxiliary variables in the form of the vector v as well as n algebraic equations.

\begin{align}
    B(x(t) \odot x(t)) \\
    \Leftrightarrow \left\{
        \begin{array}{ll}
            B(u(t) \odot x(t)) \\
            u(t) = x(t)
        \end{array}
    \right.
\end{align}

This yields the following multi linear DAE system:

\begin{align}
    \dot{x}(t) = Ax(t) + B(u(t) \odot x(t)) \\
    0 = u(t) - x(t)
\end{align}

\begin{equation}
    A =
    \begin{pmatrix}
        -0.5 & 0.1 & 0 & 0 & 0 \\
        0.5 & -0.2 & 0.1 & 0 & 0 \\
        0 & 0 & -0.3 & 0 & 0 \\
        0 & 0 & 0 & 0.4 & -0.1 \\
        0 & 0 & 0 & 0 & -0.2 \\
    \end{pmatrix},

    B = 
    \begin{pmatrix}
        0 & 0 & 0 & 0 & 0 \\
        0 & 0& 0 & 0 & 0 \\
        0 & 0 & 0 & 0 & 0 \\
        0 & 0 & 0 & 0 & -0.1 \\
        0 & 0 & 0 & 0 & 0 \\
    \end{pmatrix},

    C = 
    \begin{pmatrix}
        0 & 0.05 & 0 & 0.1 & 0 \\
        0 & 0& 0 & 0 & 0 \\
        0 & 0 & 0 & 0 & 0 \\
        0 & 0 & 0 & 0 & 0 \\
        0 & 0 & 0 & 0 & -0.1 \\
    \end{pmatrix}
\end{equation}

Finally we transform into the tenorsorized formulation of an iMTI.
Where $F, G \in \mathbb{R}^{\times_2 2n \times n}$
\begin{align}
    \dot{x}(t) &= <F|M(x,u)> \\
    0 &= <G|M(x,u)>
\end{align}


\begin{align}
    F_{i_1, ..., i_n, j_1, ...j_n, k} &= \left\{
        \begin{array}{ll}
            A_{k,l} & \text{if } i_l = 1 \text{ and } \sum_{m= 1}^n i_m + j_m = 1 \\
            B_{k,l} +  & \text{if } i_l = 1, j_l = 1 \text{ and } \sum_{m= 1}^n i_m + j_m = 2 \\
            0 & \text{otherwise} 
        \end{array}
    \right.\\
    G_{i_1, ..., i_n, j_1, ...j_n, k} &= \left\{
        \begin{array}{ll}
            1 & \text{if } i_l = 1, l=k \text{ and } \sum_{m= 1}^2 i_m + j_m = 1 \\
            -1 & \text{if } j_l = 1, l=k \text{ and } \sum_{m= 1}^2 i_m + j_m = 1 \\
            0 & \text{otherwise} 
        \end{array}
    \right.\\
\end{align}

\section{Case Study 2}

For this case study we define a non linear EDO of the following form:
Where $x \in \mathcal{C}(\mathbb{R}^n, \mathbb{R})$, $A_i \in \mathbb{R}^{n \times n } \quad \forall i$.
\begin{equation}
    \dot{x}_i(t) = x^T(t) A_i x(t) \ \ \forall i \in \{1,...,n\} 
\end{equation}

Using the same procedure done in the first case study we build the explicit multilinear tensor equations defining the model.
\begin{align}
    \dot{x}(t) &= <F|M(x,u)> \\
    0 &= <G|M(x,u)>
\end{align}

\begin{align}
    F_{i_1, ..., i_n, j_1, ...j_n, k} &= \left\{
        \begin{array}{ll}
            A_{k,l,m} + A_{k,m,l} & \text{if } l \neq m, \quad i_l = 1, i_m = 1 \text{ and } \sum_{m= 1}^n i_m + j_m = 2 \\
            A_{k,l,l} & \text{if } i_l = 1, j_l = 1 \text{ and } \sum_{m= 1}^n i_m + j_m = 2 \\
            0 & \text{otherwise} 
        \end{array}
    \right.\\
    G_{i_1, ..., i_n, j_1, ...j_n, k} &= \left\{
        \begin{array}{ll}
            1 & \text{if } i_l = 1, l=k \text{ and } \sum_{m= 1}^n i_m + j_m = 1 \\
            -1 & \text{if } j_l = 1, l=k \text{ and } \sum_{m= 1}^n i_m + j_m = 1 \\
            0 & \text{otherwise} 
        \end{array}
    \right.\\
\end{align}

Let's us now analize the DAE's index. By virtue of the equations being in explicit form the first equation are already a ODE so they are of differential index 0.
On the other hand the algbraic equations consist exclusively of equations of the form $u_i = x_i$, therefore differentiating the equation once already gives a EDO.
This means that the DAE in this form is of index 1. 

\end{document}